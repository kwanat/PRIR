\documentclass[a4paper,12pt]{article}
\usepackage{amsmath}
\usepackage{amssymb}
\usepackage[polish]{babel}
\usepackage{polski}
\usepackage[utf8]{inputenc}
\usepackage{indentfirst}
\usepackage{geometry}
\usepackage{array}
\usepackage[pdftex]{color,graphicx}
\usepackage{subfigure}
\usepackage{afterpage}
\usepackage{setspace}
\usepackage{color}
\usepackage{wrapfig}
\usepackage{listings}
\usepackage{datetime}

\renewcommand{\onehalfspacing}{\setstretch{1.6}}

\geometry{tmargin=2.5cm,bmargin=2.5cm,lmargin=2.5cm,rmargin=2.5cm}
\setlength{\parindent}{1cm}
\setlength{\parskip}{0mm}

\newenvironment{lista}{
\begin{itemize}
  \setlength{\itemsep}{1pt}
  \setlength{\parskip}{0pt}
  \setlength{\parsep}{0pt}
}{\end{itemize}}

\newcommand{\linia}{\rule{\linewidth}{0.4mm}}

\definecolor{lbcolor}{rgb}{0.95,0.95,0.95}
\lstset{
    backgroundcolor=\color{lbcolor},
    tabsize=4,
  language=c++,
  captionpos=b,
  tabsize=3,
  frame=lines,
  numbers=left,
  numberstyle=\tiny,
  numbersep=5pt,
  breaklines=true,
  showstringspaces=false,
  basicstyle=\footnotesize,
  identifierstyle=\color{magenta},
  keywordstyle=\color[rgb]{0,0,1},
  commentstyle=\color{green},
  stringstyle=\color{red}
  }

\begin{document}

\noindent
\begin{tabular}{|c|p{11cm}|c|} \hline 
Grupa 6 & Dariusz Szczupak, Kamil Wanat & \ddmmyyyydate\today \tabularnewline
\hline 
\end{tabular}


\section*{Zadanie 6 - Liczby pierwsze CUDA}

Zadanie laboratoryjne polegało na zaimplementowaniu programu sprawdzającego czy podane liczby są liczbami pierwszymi, czy złożonymi. Program zaimplementowany został z wykorzystaniem języka programowania CUDA C umożliwiającego programowanie z wykorzystaniem procesorów graficznych GPU. Problem został rozwiązany przy pomocy algorytmu ''naiwnego". Jest to metoda nieco wolniesza niż metody probabilistyczne, jednak dająca zdecydowanie lepsze wyniki. Polega ona na dzieleniu testowanego elementu przez kolejne liczby od 2 do pierwiastka kwadratowego z tejże. W naszym programie podejście naiwne zostało nieco zmodyfikowane. Zamiast sprawdzać dzielniki od 2 do pierwiastka z każdej liczby, znajdujemy najpierw liczbę maksymalną, a następnie testujemy wszystkie liczby przy pomocy dzielników od 2 do sqrt(max). Ponadto nie testujemy każdej liczby osobno dla danego dzielnika, lecz cały zbiór. 

\begin{lstlisting}
__global__ void primeTesting (number* tab, uint sqr, uint d)         
{    
uint tid=blockIdx.x;                                                                          
uint i,j;                                                                                     
for (i=2;i<=sqr;i++) {			
        for (j = tid; j <d; j+=gridDim.x) {	
                if((tab[j].value%i==0)&&(tab[j].value!=i)) 
                    tab[j].prime=false;    
        }
    }                                      
}
\end{lstlisting}


Powyższy kod przedstawia funkcję primeTesting uruchamianą na procesorze graficznym GPU. Odpowiedzialna jest ona za testowanie pierwszości liczb. Jako argument przyjmuje wskaźnik do obszaru pamięci w którym znajdują się dane liczb do testowania, pierwiastek z wartości maksymalnej liczby, oraz rozmiar danych. Każdy z wątków odpowiada za sprawdzenie części liczb domniemanie pierwszych. Liczby do sprawdzenia przez wątek definiowane są  poprzez zmienną tid - początkowo przyjmującą numer bloku, oraz zwiększaną o calkowitą ilosć bloków. Na kolejnym listingu widoczny jest fragment kopiowania danych do pamięci karty graficznej, wywołanie funkcji jądra oraz kopiowanie powrotne danych do pamięci hosta.

\begin{lstlisting}
    number* tab2;               
    number* temp = tab.data();         
    cudaMalloc( (void**)&tab2, d * sizeof(number) );                     
    cudaMemcpy(tab2, temp, d * sizeof(number), cudaMemcpyHostToDevice);  
    primeTesting <<< blockNumber, 1 >>> (tab2, sqr,d);                    
    number * result;                                                            
    result= (number *) malloc (d*sizeof(number));                               
    cudaMemcpy(result, tab2, d * sizeof(number), cudaMemcpyDeviceToHost);       
\end{lstlisting}

Poniższa tabela przedstawia zalezność przyspieszenia od ilości bloków wykorzystywanych przez program.
\begin{table}[!hbp]
\centering
\begin{tabular}{|p{5cm}|c|}
\hline 
Liczba bloków & Przyspieszenie \tabularnewline
\hline 
1	& 1\tabularnewline
2	& 1,922430365\tabularnewline
4	& 3,823976317\tabularnewline
8	& 7,628674784\tabularnewline
16 &	15,25149128\tabularnewline
32 &	30,08939713\tabularnewline
64 &	58,210549\tabularnewline
128 &	112,8851456\tabularnewline
256 &	126,3510546\tabularnewline
512 &	151,0486673\tabularnewline
1024 &	152,9736427\tabularnewline
2048 &	164,897505\tabularnewline
4096 &	164,51219\tabularnewline
8192 &	164,7069294\tabularnewline
16384	& 160,7519724\tabularnewline
\hline
\end{tabular}
\caption{Zależność przyspieszenia od ilości bloków}
\end{table}



\begin{figure}[!hbp]
	\centering
  \includegraphics[width=0.7\textwidth]{wykres.png}
  \caption{wykres przyspieszenia}
\end{figure}
Wykres przyspieszenia programu testującego liczby pierwsze pokazuje czy i jak udało się zrównoleglić działanie programu. Pierwszą różnicą widoczną w porównaniu do poprzednich zadań jest rozmiar bloków wykorzystanych do zrównoleglenia programu. W poprzednich zadaniach maksymalna liczba procesorów wynosila 12, w tym przypadku liczba ta wzrasta do ponad 16000. Przyspieszenie uzyskane wynosi ponad 150 co w porównaniu do poprzednich technologii jest porażającą wartością. Wykres zaczyna spowalniać w okolicach 500 uruchomionych równolegle bloków. Związane to jest z ograniczeniami fizycznymi architektury karty graficznej. Ponadto rozmiar użytych danych ma znaczący wpływ na ilość operacji przeprowadzonych przez dany wątek. Sprowadza się to do sytuacji w której część wywołanych wątków jest bezużyteczna podczas gdy inne nadal pracują. 
 
dane przeprowadzanych testów:
\begin{lista}
 \item wzięto pod uwagę średnią pomiarów czasów wykonania programu dla liczby wątków z przedziału od 1 do 16384. 
 program był uruchamiany dla pliku pobranego z internetu.
 \item do mierzenia czasu wykorzystano zdarzenia CUDA.
 \item testy zostały wykonane na serwerze cuda.iti.pk.edu.pl . 
 \item w czasie testów na serwerze zalogowany był jedynie użytkownik testujący.
\end{lista}


\end{document}

